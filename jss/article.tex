\documentclass[article]{jss}

%%%%%%%%%%%%%%%%%%%%%%%%%%%%%%
%% declarations for jss.cls %%%%%%%%%%%%%%%%%%%%%%%%%%%%%%%%%%%%%%%%%%
%%%%%%%%%%%%%%%%%%%%%%%%%%%%%%

%% almost as usual
\author{Paul Deveau\\Institut Curie\\PSL Research University\\INSERM U900\\Mines-ParisTech \And 
        Eric Bonnet\\Institut Curie\\PSL Research University\\INSERM U900\\Mines-ParisTech}
\title{Calculating Module Enrichment and Visualizing Data on Large-scale Molecular Maps with the R packages \pkg{ACSNMineR} and \pkg{RNaviCell}}

%% for pretty printing and a nice hypersummary also set:
\Plainauthor{Paul Deveau, Eric Bonnet} %% comma-separated
\Plaintitle{Calculating Module Enrichment and Visualizing Data on Large-scale Molecular Maps with the R packages ACSNMineR and RNaviCel} %% without formatting
\Shorttitle{Module Enrichment and Data Visualization with ACSNMineR and RNaviCell} %% a short title (if necessary)

%% an abstract and keywords
\Abstract{
  The abstract of the article.
}
\Keywords{keywords, comma-separated, not capitalized, \proglang{Java}}
\Plainkeywords{keywords, comma-separated, not capitalized, Java} %% without formatting
%% at least one keyword must be supplied

%% publication information
%% NOTE: Typically, this can be left commented and will be filled out by the technical editor
%% \Volume{50}
%% \Issue{9}
%% \Month{June}
%% \Year{2012}
%% \Submitdate{2012-06-04}
%% \Acceptdate{2012-06-04}

%% The address of (at least) one author should be given
%% in the following format:
\Address{
  Eric Bonnet\\
  Computational Systems Biology of Cancer\\
  Institut Curie\\
  26, rue d'Ulm\\
  75248 Paris, France\\
  E-mail: \email{eric.bonnet@curie.fr}\\
}
%% It is also possible to add a telephone and fax number
%% before the e-mail in the following format:
%% Telephone: +43/512/507-7103
%% Fax: +43/512/507-2851

%% for those who use Sweave please include the following line (with % symbols):
%% need no \usepackage{Sweave.sty}

%% end of declarations %%%%%%%%%%%%%%%%%%%%%%%%%%%%%%%%%%%%%%%%%%%%%%%


\begin{document}

%% include your article here, just as usual
%% Note that you should use the \pkg{}, \proglang{} and \code{} commands.

\section[Introduction]{Introduction}
Biological pathways and networks comprise sets of interactions or functional
relationships, occurring at the molecular level in living cells
\citep{adriaens2008public}.  A large body of knowledge on cellular biochemistry
is organized in publicly available repositories such as the KEGG database
\citep{kanehisa2011kegg}, Reactome \citep{croft2014reactome} and MINT
\citep{zanzoni2002mint}. All these biological databases facilitate a large
spectrum of analyses, improving our understanding of cellular systems. For
instance, it is a very common practice to cross the output of high-throughput
experiments, such as mRNA or protein expression levels, with curated biological
pathways in order to visualize changes, analyze their impact on a network and
formulate new hypotheses about biological processes. Many biologists and
computational biologists establish list of genes of interest (e.g. a list of
genes that are differentially expressed between two conditions, such as normal
vs disease) and then try to see if known biological pathways are enriched with
this list of genes. 

We have recently released the Atlas of Cancer Signalling Network (ACSN), a
web-based database which
describes signaling and regulatory molecular processes that occur in
a healthy mammalian cell but that are frequently deregulated during
cancerogenesis. ACSN is  

\bibliographystyle{jss}
\bibliography{biblio}

\end{document}
